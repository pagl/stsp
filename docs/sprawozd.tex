\documentclass{article} 
\usepackage{polski} %moze wymagac dokonfigurowania latexa, ale jest lepszy niż standardowy babel'owy [polish] 
\usepackage[utf8]{inputenc} 
\usepackage[OT4]{fontenc} 
\usepackage{graphicx,color} %include pdf's (and png's for raster graphics... avoid raster graphics!)
\usepackage{url}
\usepackage{listings}
\usepackage{enumerate}
\usepackage[pdftex,hyperfootnotes=false,pdfborder={0 0 0}]{hyperref} %za wszystkimi pakietami; pdfborder nie wszedzie tak samo zaimplementowane bo specyfikacja nieprecyzyjna; pod miktex'em po prostu nie widac wtedy ramek

% Zmiana rozmiarów strony tekstu
\addtolength{\voffset}{-1cm}
\addtolength{\hoffset}{-1cm}
\addtolength{\textwidth}{2cm}
\addtolength{\textheight}{2cm}

%bardziej zyciowe parametry sterujace rozmieszczeniem rysunkow
\renewcommand{\topfraction}{.85}
\renewcommand{\bottomfraction}{.7}
\renewcommand{\textfraction}{.15}
\renewcommand{\floatpagefraction}{.66}
\renewcommand{\dbltopfraction}{.66}
\renewcommand{\dblfloatpagefraction}{.66}
\setcounter{topnumber}{9}
\setcounter{bottomnumber}{9}
\setcounter{totalnumber}{20}
\setcounter{dbltopnumber}{9}

% własny bullet list z malymi odstepami
\newenvironment{tightlist}{
\begin{itemize}
  \setlength{\itemsep}{1pt}
  \setlength{\parskip}{0pt}
  \setlength{\parsep}{0pt}}
{\end{itemize}}

%obrazkow szukamy w nastepujacym katalogu:
\graphicspath{{pics/}}

\setlength{\parindent}{0pt}

\newcounter{nalg}[section] % defines algorithm counter for chapter-level
\renewcommand{\thenalg}{\thechapter .\arabic{nalg}} %defines appearance of the algorithm counter
\DeclareCaptionLabelFormat{algocaption}{Algorithm \thenalg} % defines a new caption label as Algorithm x.y

\lstnewenvironment{algorithm}[1][] %defines the algorithm listing environment
{   
    \refstepcounter{nalg} %increments algorithm number
    \captionsetup{labelformat=algocaption,labelsep=colon} %defines the caption setup for: it ises label format as the declared caption label above and makes label and caption text to be separated by a ':'
    \lstset{ %this is the stype
        mathescape=true,
        frame=tB,
        numbers=left, 
        numberstyle=\tiny,
        basicstyle=\scriptsize, 
        keywordstyle=\color{black}\bfseries\em,
        keywords={,input, output, return, datatype, function, in, if, else, foreach, while, begin, end, } %add the keywords you want, or load a language as Rubens explains in his comment above.
        numbers=left,
        xleftmargin=.04\textwidth,
        #1 % this is to add specific settings to an usage of this environment (for instnce, the caption and referable label)
    }
}
{}

\begin{document}

\thispagestyle{empty} %bez numeru strony

\begin{center}
{\large{Sprawozdanie z laboratorium:\\
Metaheurystyki i Obliczenia Inspirowane Biologicznie}}

\vspace{3ex}

Część I: Algorytmy optymalizacji lokalnej, problem STSP

Część II: Algorytmy optymalizacji lokalnej i globalnej, problem STSP

%Część III: Eksperyment: ... (prezentację można zrobić w LaTeX - służy do tego klasa "beamer")

\vspace{3ex}
{\footnotesize\today}

\end{center}

\vspace{10ex}

Prowadzący: dr hab.~inż. Maciej Komosiński

\vspace{5ex}

Autorzy:
\begin{tabular}{lllr}
\textbf{Patryk Gliszczyński} & inf117228 & ISWD & patryk.gliszczynski@student.put.poznan.pl \\
\textbf{Mateusz Ledzianowski} & inf117226 & ISWD & mateusz.ledzianowski@student.put.poznan.pl \\
\end{tabular}

\vspace{5ex}

Zajęcia w środy, 15:10.

\vspace{35ex}

\noindent Oświadczam/y, że~niniejsze sprawozdanie zostało przygotowane wyłącznie przez~powyższych autora/ów,
a~wszystkie elementy pochodzące z innych źródeł zostały odpowiednio zaznaczone i~są cytowane w~bibliografii.  

\newpage



\section*{Udział autorów}

Sprawozdanie zostało wykonane z wykorzystaniem szablonu dr hab. inż. Macieja Komosińskiego. \cite{MiOIB}

\subsection*{Patryk Gliszczyński}
PG zaimplementował algorytmy i przygotował środowisko badawcze, napisał skrypt do generowania wykresów.

\subsection*{Mateusz Ledzianowski}
ML zaimplementowała funkcję mierzącą czas, generującą permutację, przeprowadził eksperymenty i opisał je w sprawozdaniu.

\clearpage

\section{Symetryczny problem komiwojażera (STSP)}

%krótki opis problemu, jego zastosowań i interpretacji, złożoności – do 20 linijek uzasadnienie wyboru instancji i ich nazwy
\subsection{Opis problemu}

...

\subsection{Złożoność}

...

\subsection{Heurystyka}

...

\subsection{Wybrane instancje}

... %ML

\clearpage

\section{Optymalizacja lokalna}

%opis użytych operatorów sąsiedztwa (co najmniej jeden), wielkość sąsiedztwa
\subsection{Wstęp}

...

\subsection{Operatory sąsiedztwa}

...

\subsection{Greedy}

...

\subsection{Steepest}

... %ML

\clearpage

\section{Eksperymenty}

%porównanie działania 4 algorytmów i rodzajów sąsiedztw na wszystkich instancjach problemów
% * odległość od optimum (wg jakiej miary?), przypadek średni i najlepszy (a dla chętnych również najgorszy). Dla średnich oceniamy stabilność wyników (na wykresy średnich nanosimy odchylenia standardowe)
% * czas działania
% * jakość/czas – efektywność algorytmów (proszę zaproponować dobrą miarę)
% * GS: średnia liczba kroków algorytmu i liczba ocenionych (przejrzanych) rozwiązań
%GS – wykres: jakość rozwiązania początkowego vs. jakość rozwiązania końcowego (min. 200 powtórzeń, małe punkty) dla kilku ciekawych instancji; ciekawe to takie, które pokazują jakąś niejednorodność
%GS – zależność (wykres): liczba restartów (do >300) w multi-random vs. średnie i najlepsze z dotychczas znalezionych rozwiązań dla dwóch (max. kilku) wybranych instancji. Czy opłaca się powtarzać uruchamianie? jeśli tak, to ile razy?
%obiektywna ocena podobieństwa znajdowanych rozwiązań lokalnie optymalnych dla dwóch wybranych instancji oraz ocena ich podobieństwa do optimum globalnego (jeśli dla ATSP nie znamy globalnego, używamy najlepszego lokalnego)
\subsection{Odległość od optimum}

Z rys. \ref{fig:avg} wynika, że dla każdej instancji problemu, heurystyka i algorytmy przeszukiwania lokalnego osiągają podobne wyniki, które są kilkukrotnie lepsze od rozwiązania losowego, z którego startują zarówno Greedy, jak i Steepest.

\begin{figure}
\begin{center}
\includegraphics[width=0.8\textwidth]{graphs/algorithm_score_comparison_bar_avg.png}
\end{center}
\caption{Porównanie średnich rozwiązań na~różnych instancjach.}
\label{fig:avg}
\end{figure}

\begin{figure}
\begin{center}
\includegraphics[width=0.8\textwidth]{graphs/algorithm_score_comparison_bar_min.png}
\end{center}
\caption{Porównanie najlepszych znalezionych rozwiązań przez~algorytmy na~różnych instancjach.}
\label{fig:best}
\end{figure}

\begin{figure}
\begin{center}
\includegraphics[width=0.8\textwidth]{graphs/algorithm_score_comparison_bar_max.png}
\end{center}
\caption{Porównanie najgorszych znalezionych rozwiązań przez~algorytmy na~różnych instancjach.}
\label{fig:worst}
\end{figure}

\begin{figure}
\begin{center}
\includegraphics[width=0.8\textwidth]{graphs/algorithm_score_comparison_violin.png}
\end{center}
\caption{Porównanie rozkładów znalezionych rozwiązań przez~algorytmy na~różnych instancjach.}
\label{fig:distribution}
\end{figure}

\subsection{Czas działania}

Algorytm losowy oraz heurystyka są zdecydowanie najszybsze, ponieważ każde z nich sprawdza tylko jedno rozwiązanie. Greedy i Steepest przeszukują przestrzeń rozwiązań, dopóki nie mogą już bardziej poprawić wyniku, co zajmuje znacznie więcej czasu.

\begin{figure}
\begin{center}
%\includegraphics[width=0.8\textwidth]{graphs/rys_time.pdf}
\end{center}
\caption{Porównanie czasu działania algorytmów na~poszczególnych instancjach.}
\label{fig:best}
\end{figure}

\subsection{Efektywność algorytmów}

\subsubsection{Wybrana miara}

Aby porównać algorytmy pod względem jakości, można to zrobić przez zdefiniowanie kosztu czasowego, jaki trzeba ponieść, aby uzyskać dane rozwiązanie. Czyli należy policzyć iloraz $cost = time / result$, co przedstawia rys. \ref{fig:cost}. Natomiast efektywnością algorytmu jest odwrotność kosztu, która została przedstawiona na rys. \ref{fig:quality}.

\subsubsection{Wyniki}

Z wykresów \ref{fig:cost} i \ref{fig:quality} można by wyciągnąć wniosek, że najefektywniejszym algorytmem są losowy i heurystyka, ponieważ zajmują najmniej czasu. Nie można przy tym jednak zapominać... %TODO

\begin{figure}
\begin{center}
%\includegraphics[width=0.8\textwidth]{graphs/rys_efficiency.pdf}
\end{center}
\caption{Porównanie kosztów algorytmów na~poszczególnych instancjach.}
\label{fig:cost}
\end{figure}

\begin{figure}
\begin{center}
%\includegraphics[width=0.8\textwidth]{graphs/rys_efficiency.pdf}
\end{center}
\caption{Porównanie efektywności algorytmów na~poszczególnych instancjach.}
\label{fig:quality}
\end{figure}

\subsection{Średnia liczba kroków}

...

\begin{figure}
\begin{center}
%\includegraphics[width=0.8\textwidth]{graphs/rys_steps.pdf}
\end{center}
\caption{Porównanie algorytmów Greedy Search i~Steepest pod~względem liczby kroków do~zatrzymania.}
\label{fig:steps}
\end{figure}

\subsection{Średnia liczba przeszukanych rozwiązań}

...

\begin{figure}
\begin{center}
%\includegraphics[width=0.8\textwidth]{graphs/rys_n_solutions.pdf}
\end{center}
\caption{Porównanie algorytmów Greedy Search i~Steepest pod~względem liczby przeszukanych rozwiązań.}
\label{fig:nsol}
\end{figure}

\subsection{Greedy Search}

\subsubsection{Jakość rozwiązania początkowego a końcowego}

...

\begin{figure}
\begin{center}
%\includegraphics[width=0.8\textwidth]{graphs/rys_difference_start_final.pdf}
\end{center}
\caption{Porównanie jakości rozwiązań początkowych i~końcowych przez~algorytmy Greedy Search i~Steepest.}
\label{fig:diff}
\end{figure}

\subsubsection{Wielokrotne uruchamianie dla różnych rozwiązań początkowych}

...

\begin{figure}
\begin{center}
%\includegraphics[width=0.8\textwidth]{graphs/rys_more_starts.pdf}
\end{center}
\caption{Porównanie jakości rozwiązań algorytmów Gready Search i~Steepest w~zależności od~liczby uruchomień tych algorytmów dla~różnych rozwiązań początkowych.}
\label{fig:more}
\end{figure}

\subsection{Porównanie rozwiązań}

\subsubsection{Miara odległości rozwiązań od rozwiązania optymalnego}

...

\subsubsection{Wyniki}

...

\begin{figure}
\begin{center}
%\includegraphics[width=0.8\textwidth]{graphs/rys_opt_distance.pdf}
\end{center}
\caption{Porównanie odległości znajdowanych rozwiązań przez algorytmy od rozwiązania optymalnego.}
\label{fig:dist}
\end{figure} %ML

\clearpage

\section{Podsumowanie}

%wnioski (od ogólnych do szczegółowych) z przeprowadzonych doświadczeń
%trudności na jakie napotkano
%uzasadnienie wprowadzanych ulepszeń, propozycje udoskonaleń i ich spodziewane efekty
\subsection{Wnioski}

...

\subsection{Trudności}

...

\subsection{Propozycje udoskonaleń} %ML

\clearpage

%%%%%%%%%%%%%%%% literatura %%%%%%%%%%%%%%%%

\bibliography{sprawozd}
\bibliographystyle{plain}

\end{document}