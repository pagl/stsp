\documentclass{article} 
\usepackage{polski} 
\usepackage[utf8]{inputenc} 
\usepackage[OT4]{fontenc} 
\usepackage{graphicx,color}
\usepackage{enumerate}
\usepackage{csvsimple}
\usepackage{listings}
\usepackage{caption}
\usepackage{float}
\usepackage{color}
\usepackage{url}
\usepackage[pdftex,hyperfootnotes=false,pdfborder={0 0 0}]{hyperref}

\definecolor{part2}{rgb}{0.1, 0.1, 0.8}

% Zmiana rozmiarów strony tekstu
\addtolength{\voffset}{-1cm}
\addtolength{\hoffset}{-1cm}
\addtolength{\textwidth}{2cm}
\addtolength{\textheight}{2cm}

%bardziej zyciowe parametry sterujace rozmieszczeniem rysunkow
\renewcommand{\topfraction}{.85}
\renewcommand{\bottomfraction}{.7}
\renewcommand{\textfraction}{.15}
\renewcommand{\floatpagefraction}{.66}
\renewcommand{\dbltopfraction}{.66}
\renewcommand{\dblfloatpagefraction}{.66}
\setcounter{topnumber}{9}
\setcounter{bottomnumber}{9}
\setcounter{totalnumber}{20}
\setcounter{dbltopnumber}{9}

% własny bullet list z malymi odstepami
\newenvironment{tightlist}{
\begin{itemize}
  \setlength{\itemsep}{1pt}
  \setlength{\parskip}{0pt}
  \setlength{\parsep}{0pt}}
{\end{itemize}}

%obrazkow szukamy w nastepujacym katalogu:
\graphicspath{{pics/}}

\setlength{\parindent}{0pt}

\newcounter{nalg}[section] % defines algorithm counter for chapter-level
\renewcommand{\thenalg}{\thechapter .\arabic{nalg}} %defines appearance of the algorithm counter
\DeclareCaptionLabelFormat{algocaption}{Algorithm \thenalg} % defines a new caption label as Algorithm x.y

\lstnewenvironment{algorithm}[1][] %defines the algorithm listing environment
{   
    \refstepcounter{nalg} %increments algorithm number
    \captionsetup{labelformat=algocaption,labelsep=colon} %defines the caption setup for: it ises label format as the declared caption label above and makes label and caption text to be separated by a ':'
    \lstset{ %this is the stype
        mathescape=true,
        frame=tB,
        numbers=left, 
        numberstyle=\tiny,
        basicstyle=\scriptsize, 
        keywordstyle=\color{black}\bfseries\em,
        keywords={,input, output, return, datatype, function, in, if, else, foreach, while, begin, end, } %add the keywords you want, or load a language as Rubens explains in his comment above.
        numbers=left,
        xleftmargin=.04\textwidth,
        #1 % this is to add specific settings to an usage of this environment (for instnce, the caption and referable label)
    }
}
{}

\begin{document}

\sloppy
\widowpenalty10000
\clubpenalty10000

\thispagestyle{empty} %bez numeru strony

\begin{center}
{\large{Sprawozdanie z laboratorium:\\
Metaheurystyki i Obliczenia Inspirowane Biologicznie}}

\vspace{3ex}

Część I: Algorytmy optymalizacji lokalnej, problem STSP

Część II: Algorytmy optymalizacji lokalnej i globalnej, problem STSP

%Część III: Eksperyment: ... (prezentację można zrobić w LaTeX - służy do tego klasa "beamer")

\vspace{3ex}
{\footnotesize\today}

\end{center}

\vspace{10ex}

Prowadzący: dr hab.~inż. Maciej Komosiński

\vspace{5ex}

Autorzy:
\begin{tabular}{lllr}
\textbf{Patryk Gliszczyński} & inf117228 & ISWD & patryk.gliszczynski@student.put.poznan.pl \\
\textbf{Mateusz Ledzianowski} & inf117226 & ISWD & mateusz.ledzianowski@student.put.poznan.pl \\
\end{tabular}

\vspace{5ex}

Zajęcia w środy, 15:10.

\vspace{35ex}

\noindent Oświadczam/y, że~niniejsze sprawozdanie zostało przygotowane wyłącznie przez~powyższych autora/ów,
a~wszystkie elementy pochodzące z innych źródeł zostały odpowiednio zaznaczone i~są cytowane w~bibliografii.  

\newpage



\section*{Udział autorów}

Sprawozdanie zostało wykonane z wykorzystaniem szablonu dr hab. inż. Macieja Komosińskiego \cite{MiOIB}.

\subsection*{Patryk Gliszczyński}
PG~zaimplementował algorytmy lokalnego przeszukiwania, przygotował środowisko badawcze, napisał skrypt do~automatycznego generowania wykresów, oraz~opisał teoretyczną stronę tego~zagadnienia.

PG~zaimplementował i~opisał algorytm symulowanego wyżarzania.

\subsection*{Mateusz Ledzianowski}
ML~zaimplementował funkcję mierzącą czas, algorytm losowy i~prostą heurystykę, przeprowadził eksperymenty na~różnych instancjach problemu, oraz~opisał wnioski i~spostrzeżenia z~przeprowadzonych badań.

ML~zaimplementował i~opisał algorytm przeszukiwania tabu w wersji master list.

\clearpage

\section{Symetryczny problem komiwojażera (STSP)}

\subsection{Opis problemu}

...

\subsection{Złożoność}

...

\subsection{Heurystyka}

...

\subsection{Wybrane instancje}

... %ML

\clearpage

\section{Optymalizacja lokalna}

%opis użytych operatorów sąsiedztwa (co najmniej jeden), wielkość sąsiedztwa
\subsection{Wstęp}

...

\subsection{Operatory sąsiedztwa}

...

\subsection{Greedy}

...

\subsection{Steepest}

... %ML

\clearpage
{\color{part2}
\section{Symulowane wyżarzanie}

\subsection{Opis}

Algorytm symulowanego wyżarzania jest~heurystyczną metodą przeszukującą przestrzeń alternatywnych rozwiązań w~celu znalezienia najlepszego rozwiązania, czyli~takiego będącego najbliżej optimum pod~względem wartości funkcji kosztu. Polega on~na~iteracyjnym wyborze losowych rozwiązań z~przestrzeni lokalnych solucji i~akceptacji ich w~przypadku gdy~zmniejszają one~wartość funkcji kosztu, bądź~spełniają odpowiednie warunki probabilistyczne degradowane wraz z~upływem kolejnych iteracji. Algorytm opisujący kolejne kroki tej~metody został przedstawiony poniżej.

\begin{enumerate}
	\item Wygeneruj temperaturę startową $T_0$, rozwiązanie początkowe $x = x_{init}$, liczbę kroków $L$ co~które~aktualizowana jest~temperatura, oraz~stopień redukcji temperatury $\alpha~\epsilon~(0, 1)$.
	\item Powtarzaj dopóki warunek stopu nie~został spełniony:
	\begin{enumerate}
		\item Powtarzaj $L$ razy:
		\begin{enumerate}
			\item Generuj losowe rozwiązanie $x'$ w~sąsiedztwie rozwiązania $x$.
			\item Jeżeli nowe rozwiązanie jest~lepsze od poprzedniego $f(x') <= f(x)$, lub~warunek wyboru gorszego rozwiązania zostanie spełniony:
			\begin{enumerate}
				\item Zaakceptuj nowe rozwiązanie $x = x'$.
			\end{enumerate}
		\end{enumerate}
		\item Aktualizuj temperaturę $T_k = T_{k-1} * \alpha$.
	\end{enumerate}
\end{enumerate}

Powyżej przedstawiony generyczny algorytm może zostać wykorzystany w~wielu scenariuszach optymalizacyjnych. Dla~zadania postawionego w~tym zadaniu należało odpowiednio dostroić powyższy algorytm, tak~aby~najskuteczniej radził sobie w~znajdowaniu rozwiązania dla~problemu STSP.


\subsection{Dobór temperatury początkowej}
Jednym z~ważniejszych kroków w~procesie symulowanego wyżarzania jest~dobór odpowiedniej temperatury początkowej. Wartość ta~decyduje w~dużej mierze o~czasie zbieżności do~rozwiązania finalnego oraz~liczbie zaakceptowanych gorszych lokalnie rozwiązań, które~mogą prowadzić do~nieoczekiwanej poprawy jakości rozwiązania w~ujęciu globalnym. W~naszej implementacji zdecydowaliśmy się dobierać ten parametr w~sposób adaptatywny, ściśle zależny od~instancji problemu oraz~wygenerowanego pierwszego rozwiązania. Wartość ta~wyliczana jest~zgodnie z~poniższym wzorem.

\begin{equation}
	T_0 = \frac{\sum_{k=0}^{K} f(G(x)) - f(x)}{K * \ln \theta}
\end{equation}

W~powyższym wzorze, wektor $x$ jest~oryginalnym wygenerowanym rozwiązaniem, funkcja $G(x)$ generuje nowe rozwiązanie $x'$ w~sąsiedztwie rozwiązania $x$, a~funkcja $f(x)$ zwraca wartość kosztu danego rozwiązania. Parametr $K$ określa liczbę rozwiązań, które~chcemy wygenerować, a~$\theta$ procent akceptowanych ruchów. Tak~przedstawiona metoda pozwala na~lepszy dobór temperatury początkowej w~odniesieniu do~konkretnej instancji problemu. W~postaci przedstawionej powyżej jest~ona jednak wrażliwa na~jakość wygenerowanego rozwiązania początkowego i~może nie~zawsze być~prawidłowo dobrana. Wartość ta~może być~również obliczana rekurencyjnie biorąc pod~uwagę szersze spektrum rozwiązań co~pozwala na dokładniejszy dobór parametru \cite{saTemperature}. 


\subsection{Generowanie nowego rozwiązania}
W~każdej iteracji algorytmu symulowanego wyżarzania należy wybrać jedno losowe rozwiązanie z~przestrzeni lokalnych rozwiązań w~sąsiedztwie OPT-2. W~tym celu zdecydowaliśmy się wylosować dwie różne liczby całkowite w~przedziale $(0, n)$, gdzie~$n$ jest~rozmiarem wektora rozwiązania, a~następnie dokonać odwrócenia łuku między odpowiadającymi wierzchołkami w~wektorze rozwiązania. Dzięki takiemu podejściu, nie~jest~potrzebne marnotrawienie zasobów operacyjnych i~pamięciowych na~generowanie i~przechowywanie wszystkich rozwiązań dla~danej iteracji algorytmu, z~których następnie wybrane zostałoby tylko pojedyncze rozwiązanie, a~reszta byłaby zbędna.


\subsection{Warunek akceptacji rozwiązania}
Rozwiązanie $x'$ generowane w~każdej iteracji algorytmu zostaje zaakceptowane tylko w~przypadku, gdy~jest~ono nie~gorsze od~rozwiązania poprzedniego $f(x') <= f(x)$, lub~spełniony jest~warunek probabilistyczny:

\begin{equation}
	e^{-\frac{f(x') - f(x)}{T_k}} > random(0, 1)
\end{equation} 

Dzięki takiemu podejściu, pozwalamy algorytmowi wybrać rozwiązanie gorsze z~pewnym prawdopodobieństwem zależnym od~aktualnej wartości temperatury. Może to~mieć pozytywny skutek i~pozwolić nam wyjść z~lokalnego optimum, w~którym moglibyśmy utknąć gdybyśmy heurystycznie dążyli tylko ku~lokalnej poprawie. Należy zauważyć, iż~poprzez degradacje wartości temperatury w~kolejnych iteracjach, prawdopodobieństwo wybrania gorszego rozwiązania maleje.


\subsection{Warunek stopu}
Istotą każdego algorytmu optymalizacyjnego jest~dobrze zdefiniowany warunek stopu. W~naszej implementacji zdecydowaliśmy się kończyć przeszukiwanie przestrzeni rozwiązań, w~przypadku gdy~wartość temperatury $T_k$ spadła poniżej minimalnej wartości temperatury $T_\omega$, lub~gdy~w~przeciągu pewnej liczby iteracji $\sigma$ nie~udało się poprawić jakości rozwiązania. 


\subsection{Parametry}
Jakość algorytmu symulowanego wyżarzania zależy od~wielu parametrów, które decydują o~czasie zbieżności do~rozwiązania finalnego, liczbie przejrzanych lokalnie rozwiązań, czy~szansie na~akceptację rozwiązań gorszych. Metodą prób, błędów i~analizy dobraliśmy zestaw parametrów, które pozwalają w~relatywnie krótkim czasie znajdować rozwiązania bliskie optimum. Poniżej przedstawione zostały wartości parametrów przedstawionych w~tej sekcji.

\begin{itemize}
	\item[--] $\alpha = 0.9$ -- stopień redukcji temperatury w~kolejnych iteracjach.
	\item[--] $\theta = 0.95$ -- procent akceptowanych ruchów przy doborze temperatury początkowej.
	\item[--] $T_\omega = 0.1$ -- minimalna wartość temperatury kończąca algorytm.
	\item[--] $K = 1000$ -- liczba powtórzeń generowania rozwiązania przy~doborze temperatury.
\end{itemize}




 %PG

\clearpage

\section{Przeszukiwanie tabu}

\subsection{Opis}

Tabu Search jest drugim algorytmem optymalizacyjnym przedstawionym w sprawozdaniu, który pozwala na wychodzenie z optimów lokalnych poprzez dopuszczenie ruchów pogarszających rozwiązanie. Istnieją dwie popularne wersje tego rozwiązania -- z listą tabu, czyli zakazanych ruchów oraz z elitarną listą kandydatów. Na potrzeby eksperymentu zdecydowaliśmy się zaimplementować tę drugą wersję.

Algorytm Tabu Search jest można opisać w następujący sposób\cite{TS}:

\begin{algorithm}[caption={Integer division.}, label={alg1}]
procedure PRZESZUKIWANIE TABU
begin
  INICJALIZUJ(xstart, xbest, T)
  x := xstart
  repeat
    GENERUJ(V $\in$ N(x))
    WYBIERZ(x0) //najlepsze f w V + aspiracja
    UAKTUALNIJ LISTE TABU(T)
    if f (x0) $\le$ f(xbest) then xbest := x0
    x := x0
  until WARUNEK STOPU
end
\end{algorithm}

Po zainicjalizowaniu parametrów początkowych, należy powtarzać aż do osiągnięcia warunku stopu, którym w naszym rozwiązaniu jest określona liczba iteracji bez poprawy rozwiązania, kolejne kroki: najpierw należy ustalić listę kandydatów na ruch, następnie wybrać z niej ten, który prowadzi do najlepszego rozwiązania, usunąć wybrany krok z listy kandydatów, aby go nie powtórzyć w przyszłości, na końcu uaktualnić najlepsze znalezione dotychczas rozwiązanie, jeśli aktualne jest lepsze od dotychczas najlepszego.

\subsection{Elitarna lista kandydatów}

W naszym rozwiązaniu postanowiliśmy zastosować elitarną listę kandydatów. Polega ona na tym, że w momencie, kiedy dotychczas posiadana lista kandydatów jest niewystarczająca, co może się zdarzyć z dwóch powodów -- jest pusta lub najlepszy kandydat na liście jest gorszy od najsłabszego kandydata w momencie generowania tej listy, należy stworzyć nową listę kandydatów.

Aby stworzyć nową listę kandydatów, należy przejrzeć całe sąsiedztwo aktualnego rozwiązania i wybrać spośród sąsiadów określoną listę najlepszych. Ruchy, które prowadzą do ich osiągnięcia tworzą nową elitarną listę kandydatów.

W następnych krokach algorytmu lista ta jest utrzymywana i sprawdza się tylko tych sąsiadów aktualnego rozwiązania, do których prowadzą ruchy z listy i wybiera się spośród nich najlepszy. Następnie ruch ten jest z listy usuwany, aby nie został powtórzony w przyszłości, aż do czasu wygenerowania nowej elitarnej listy kandydatów.

Dzięki takiemu rozwiązaniu, można znacznie ograniczyć liczbę przeszukiwanych sąsiadów każdego rozwiązania i wychodzić z optimów lokalnych -- gdy nie ma rozwiązania, które poprawia wynik, też tworzy się listy kandydatów, które pogarszają go w najmniejszym stopniu.

\subsection{Cechy algorytmu}

Jedną z głównych cech algorytmu i zarazem przyczyn jego powstania jest dopuszczenie wychodzenia z maksimów lokalnych. Algorytm ten pozwala także na ograniczenie się do przeszukiwania jedynie części sąsiadów, zamiast całego ich zbioru, co znacznie przyspiesza obliczenia. Jednak aby możliwe było pokonanie słabości algorytmów optymalizacji lokalnej, przeszukiwanie tabu naraża się na wpadanie w cykle. Trzeba też uważać, aby nie ograniczyć za bardzo jego ruchów.

\subsection{Parametry}

Nasza implementacja tabu search ma zasadniczo dwa główne parametry -- wielkość elitarnej listy kandydatów $k$ i liczba iteracji bez poprawy najlepszego rozwiązania do zakończenia algorytmu $P \cdot L$.
Początkowo wyszliśmy od parametrów $k = sizeofinstance() / 10$ i $P \cdot L = 10 \cdot sizeofinstance()$. Próbowaliśmy je ręcznie zmieniać i konfigurować, jednak nie poprawiało to znacząco wyników. Zauważyliśmy jedynie, że zmniejszając $k$ lub $L \cdot P$, algorytm szybciej się kończył, często jednak z gorszym rezultatem.

\subsection{Podsumowanie}

Algorytm przeszukiwania Tabu pozwala przezwyciężyć słabości algorytmów przeszukiwania lokalnego, poprzez wychodzenie z maksimów lokalnych i zapamiętywania dotychczasowego najlepszego rozwiązania. Ponadto w czasie swego działania sprawdza on dużo więcej rozwiązań, jednocześnie działając dłużej. Szybciej jednak podejmuje decyzję o wyborze następnego rozwiązania niż algorytm steepest, ponieważ nie przeszukuje całego sąsiedztwa. %ML
}
\clearpage

\section{Eksperymenty}

%porównanie działania 4 algorytmów i rodzajów sąsiedztw na wszystkich instancjach problemów
% * odległość od optimum (wg jakiej miary?), przypadek średni i najlepszy (a dla chętnych również najgorszy). Dla średnich oceniamy stabilność wyników (na wykresy średnich nanosimy odchylenia standardowe)
% * czas działania
% * jakość/czas – efektywność algorytmów (proszę zaproponować dobrą miarę)
% * GS: średnia liczba kroków algorytmu i liczba ocenionych (przejrzanych) rozwiązań
%GS – wykres: jakość rozwiązania początkowego vs. jakość rozwiązania końcowego (min. 200 powtórzeń, małe punkty) dla kilku ciekawych instancji; ciekawe to takie, które pokazują jakąś niejednorodność
%GS – zależność (wykres): liczba restartów (do >300) w multi-random vs. średnie i najlepsze z dotychczas znalezionych rozwiązań dla dwóch (max. kilku) wybranych instancji. Czy opłaca się powtarzać uruchamianie? jeśli tak, to ile razy?
%obiektywna ocena podobieństwa znajdowanych rozwiązań lokalnie optymalnych dla dwóch wybranych instancji oraz ocena ich podobieństwa do optimum globalnego (jeśli dla ATSP nie znamy globalnego, używamy najlepszego lokalnego)
\subsection{Odległość od optimum}

Z rys. \ref{fig:avg} wynika, że dla każdej instancji problemu, heurystyka i algorytmy przeszukiwania lokalnego osiągają podobne wyniki, które są kilkukrotnie lepsze od rozwiązania losowego, z którego startują zarówno Greedy, jak i Steepest.

\begin{figure}
\begin{center}
\includegraphics[width=0.8\textwidth]{graphs/algorithm_score_comparison_bar_avg.png}
\end{center}
\caption{Porównanie średnich rozwiązań na~różnych instancjach.}
\label{fig:avg}
\end{figure}

\begin{figure}
\begin{center}
\includegraphics[width=0.8\textwidth]{graphs/algorithm_score_comparison_bar_min.png}
\end{center}
\caption{Porównanie najlepszych znalezionych rozwiązań przez~algorytmy na~różnych instancjach.}
\label{fig:best}
\end{figure}

\begin{figure}
\begin{center}
\includegraphics[width=0.8\textwidth]{graphs/algorithm_score_comparison_bar_max.png}
\end{center}
\caption{Porównanie najgorszych znalezionych rozwiązań przez~algorytmy na~różnych instancjach.}
\label{fig:worst}
\end{figure}

\begin{figure}
\begin{center}
\includegraphics[width=0.8\textwidth]{graphs/algorithm_score_comparison_violin.png}
\end{center}
\caption{Porównanie rozkładów znalezionych rozwiązań przez~algorytmy na~różnych instancjach.}
\label{fig:distribution}
\end{figure}

\subsection{Czas działania}

Algorytm losowy oraz heurystyka są zdecydowanie najszybsze, ponieważ każde z nich sprawdza tylko jedno rozwiązanie. Greedy i Steepest przeszukują przestrzeń rozwiązań, dopóki nie mogą już bardziej poprawić wyniku, co zajmuje znacznie więcej czasu.

\begin{figure}
\begin{center}
%\includegraphics[width=0.8\textwidth]{graphs/rys_time.pdf}
\end{center}
\caption{Porównanie czasu działania algorytmów na~poszczególnych instancjach.}
\label{fig:best}
\end{figure}

\subsection{Efektywność algorytmów}

\subsubsection{Wybrana miara}

Aby porównać algorytmy pod względem jakości, można to zrobić przez zdefiniowanie kosztu czasowego, jaki trzeba ponieść, aby uzyskać dane rozwiązanie. Czyli należy policzyć iloraz $cost = time / result$, co przedstawia rys. \ref{fig:cost}. Natomiast efektywnością algorytmu jest odwrotność kosztu, która została przedstawiona na rys. \ref{fig:quality}.

\subsubsection{Wyniki}

Z wykresów \ref{fig:cost} i \ref{fig:quality} można by wyciągnąć wniosek, że najefektywniejszym algorytmem są losowy i heurystyka, ponieważ zajmują najmniej czasu. Nie można przy tym jednak zapominać... %TODO

\begin{figure}
\begin{center}
%\includegraphics[width=0.8\textwidth]{graphs/rys_efficiency.pdf}
\end{center}
\caption{Porównanie kosztów algorytmów na~poszczególnych instancjach.}
\label{fig:cost}
\end{figure}

\begin{figure}
\begin{center}
%\includegraphics[width=0.8\textwidth]{graphs/rys_efficiency.pdf}
\end{center}
\caption{Porównanie efektywności algorytmów na~poszczególnych instancjach.}
\label{fig:quality}
\end{figure}

\subsection{Średnia liczba kroków}

...

\begin{figure}
\begin{center}
%\includegraphics[width=0.8\textwidth]{graphs/rys_steps.pdf}
\end{center}
\caption{Porównanie algorytmów Greedy Search i~Steepest pod~względem liczby kroków do~zatrzymania.}
\label{fig:steps}
\end{figure}

\subsection{Średnia liczba przeszukanych rozwiązań}

...

\begin{figure}
\begin{center}
%\includegraphics[width=0.8\textwidth]{graphs/rys_n_solutions.pdf}
\end{center}
\caption{Porównanie algorytmów Greedy Search i~Steepest pod~względem liczby przeszukanych rozwiązań.}
\label{fig:nsol}
\end{figure}

\subsection{Greedy Search}

\subsubsection{Jakość rozwiązania początkowego a końcowego}

...

\begin{figure}
\begin{center}
%\includegraphics[width=0.8\textwidth]{graphs/rys_difference_start_final.pdf}
\end{center}
\caption{Porównanie jakości rozwiązań początkowych i~końcowych przez~algorytmy Greedy Search i~Steepest.}
\label{fig:diff}
\end{figure}

\subsubsection{Wielokrotne uruchamianie dla różnych rozwiązań początkowych}

...

\begin{figure}
\begin{center}
%\includegraphics[width=0.8\textwidth]{graphs/rys_more_starts.pdf}
\end{center}
\caption{Porównanie jakości rozwiązań algorytmów Gready Search i~Steepest w~zależności od~liczby uruchomień tych algorytmów dla~różnych rozwiązań początkowych.}
\label{fig:more}
\end{figure}

\subsection{Porównanie rozwiązań}

\subsubsection{Miara odległości rozwiązań od rozwiązania optymalnego}

...

\subsubsection{Wyniki}

...

\begin{figure}
\begin{center}
%\includegraphics[width=0.8\textwidth]{graphs/rys_opt_distance.pdf}
\end{center}
\caption{Porównanie odległości znajdowanych rozwiązań przez algorytmy od rozwiązania optymalnego.}
\label{fig:dist}
\end{figure} %ML

\clearpage

\section{Podsumowanie}

%wnioski (od ogólnych do szczegółowych) z przeprowadzonych doświadczeń
%trudności na jakie napotkano
%uzasadnienie wprowadzanych ulepszeń, propozycje udoskonaleń i ich spodziewane efekty
\subsection{Wnioski}

...

\subsection{Trudności}

...

\subsection{Propozycje udoskonaleń} %ML

\clearpage

%%%%%%%%%%%%%%%% literatura %%%%%%%%%%%%%%%%

\bibliography{sprawozd}
\bibliographystyle{plain}

\end{document}