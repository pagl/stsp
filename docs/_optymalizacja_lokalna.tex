\subsection{Wstęp}

Algorytmy optymalizacji lokalnej pozwalają na poszukiwanie lepszych rozwiązań od aktualnie znalezionego, tak długo, aż w zadanym sąsiedztwie nie można go już bardziej poprawić. Dzięki takiej optymalizacji, nie trzeba przeszukiwać całej przestrzeni rozwiązań, a osiągać dobre wyniki. 

\subsection{Operatory sąsiedztwa}

Aby stosować przeszukiwanie lokalne, należy określić sąsiedztwo dla każdego rozwiązania. Jest~to~przestrzeń rozwiązań, które~są~podobne do~posiadanego. Jednym z~możliwych sąsiedztw jest~OPT-2.

\subsubsection{OPT-2}

Sąsiedztwo OPT-2 definiuje się poprzez zamianę kolejności wierzchołków na~dwóch pozycjach lub~odwrócenie kolejności odwiedzania miast na~łuku pomiędzy dwoma pozycjami. W~zależności od~tego, czy~rozważany problem komiwojażera jest~symetryczny, czy~nie --- różne sąsiedztwa sprawdzają się z~odmienną skutecznością. Rozważając, które z~sąsiedztw wybrać, należy zastanowić się, które z~nich w~najmniejszym stopniu wpływa na~funkcję celu.

\paragraph{Zamiana miast}

Sąsiedztwo OPT-2 polegające na zamianie miast można zaimplementować poprzez prostą zamianę elementów na pozycji i-tej i j-tej. Jest ono skuteczniejsze dla problemu asymetrycznego komiwojażera, ponieważ zamiana łuków spowodowałaby większą różnicę w~funkcji celu --- inne byłyby nie tylko cztery drogi, prowadzące do dwóch zamienianych miast, lecz także wszystkie drogi na łuku.

\paragraph{Odwrócenie łuku}

W przypadku odwrócenia łuku pomiędzy miastami na pozycjach i-tej i~j-tej, funkcja celu ulegnie zmianie spowodowanej tylko modyfikacją dwóch dróg prowadzących do~miast i-tego i j-tego z~jednej strony, a~z~drugiej odwrócenie łuku, gdy~drogi w~obie strony mają taką samą długość, niczego nie~zmieni.

\subsection{Greedy}

Jednym z~algorytmów przeszukiwania lokalnego jest~algorytm Greedy. Polega on~na~tym, że~jeśli w~momencie przeszukiwania sąsiedztwa aktualnego rozwiązania, znajdzie rozwiązanie lepsze, natychmiast je~wybiera jako~aktualne i~szuka następnego w~jego sąsiedztwie, dopóki istnieją rozwiązania poprawiające aktualne.

\subsection{Steepest}

Algorytm Steepest różni się od~poprzednika tym, że~gdy~przegląda swoje sąsiedztwo, nie~wybiera jako~aktualne rozwiązanie, pierwszego znalezionego, lepszego rozwiązania, ale~zawsze przegląda wszystkich sąsiadów i~wybiera najlepszego spośród nich, dzięki czemu porusza się ,,po najbardziej stromym zboczu'' na~wykresie funkcji celu.