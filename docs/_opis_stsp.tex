\subsection{Opis problemu}

Problem komiwojażera modeluje sytuację znaną ze świata rzeczywistego, w której pewien obiekt stara się odwiedzić wszystkie punkty z danego zbioru punktów oraz wrócić do miejsca początkowego, w kolejności minimalizującej całkowity koszt podróży. Z tego typu zadaniem mierzą się między innymi wszelkiego rodzaju dostawcy usług transportowych, listonosze, czy~akwizytorzy. W symetrycznym problemie komiwojażera koszt pomiędzy dowolnymi dwoma wierzchołkami w grafie połączeń jest identyczny w obydwie strony. W naszym problemie sieć połączeń pomiędzy poszczególnymi wierzchołkami opisana jest grafem pełnym, co oznacza że istnieje połączenie między każdą parą wierzchołków w grafie.

\subsection{Złożoność}

W~tak postawionym problemie istnieje $n!$~różnych rozwiązań, gdzie~$n$ oznacza liczbę wierzchołków w~grafie. Punkty możemy odwiedzać w~dowolnej kolejności, zatem jeżeli zostaną one ponumerowane od~$1$ do~$n$, to~każda permutacja $n$-elementowa może~reprezentować pełne rozwiązanie. Rozwiązanie w~postaci permutacji możemy odczytywać w~taki~sposób, że~z~miejsca na~pozycji $i$, przemieszczamy się do~miejsca na~pozycji $i+1$, pamiętając o~tym, żeby~z~miejsca na~pozycji $n$ wrócić do~punktu startowego. Przestrzeń rozwiązań jest bardzo duża, oraz~kosztowna obliczeniowo przy~chęci wyznaczenia każdej istniejącej permutacji. Jeśli bylibyśmy w~stanie sprawdzać $1\ 000\ 000\ 000$ rozwiązań w~czasie $1$ sekundy, rozwiązania dokładnego dla~$n=16$, szukalibyśmy przez ok.~$6$h, a~znalezienie go~dla~$n=20$ zajęłoby $77$~lat.

\subsection{Rozwiązanie losowe}

Ponieważ rozwiązanie można reprezentować w~postaci permutacji, możliwym jest~wygenerowanie rozwiązania losowego poprzez przeprowadzenie losowej permutacji wierzchołków z~rozwiązania początkowego. Złożoność obliczeniowa takiej operacji wynosi $\theta(n)$, gdzie~$n$ jest~długością rozwiązania początkowego. Aby~wygenerować losową permutację należy zastosować poniższą procedurę:

\begin{enumerate}
    \item Wypełnij tablicę liczbami od $1$ do $n$.
    \item $i := n$. -- zainicjuj zmienną wskazującą na ostatni element rozwiązania.
    \item $random := random(0,\ i - 1) $ -- wylosuj liczbę z zakresu od $0$ do $i - 1$ włącznie.
    \item Zamień element z pozycji $i-1$ z elementem na pozycji wskazywanej przez zmienną $random$.
    \item $i := i-1$.
    \item Jeżeli $i>1$, wróć do kroku 3.
\end{enumerate}

Aby algorytmowi losowemu dać większe szanse na znalezienie lepszego rozwiązania, powtarzamy go $1\ 000\ 000$ razy i zwraca on najlepsze spośród znalezionych rozwiązań.

\subsection{Heurystyka}

Dla~symetrycznego problemu komiwojażera możliwe jest~znalezienie prostej heurystyki o~złożoności $\theta(n^2)$ dającej satysfakcjonujące rezultaty, przeciętnie odległe od~rozwiązania optymalnego o~$10-15\%$~\cite{Heuristic}. W~celu wyznaczenia rozwiązania z~wykorzystaniem algorytmu heurystycznego, należy postępować zgodnie z~następującymi krokami:

\newpage

\begin{enumerate}
    \item Wybierz losowy wierzchołek startowy $x$ i~dodaj go~do~rozwiązania.
    \item Znajdź najbliższy nieodwiedzony wierzchołek $y$ względem punktu $x$.
  	\item Dodaj wierzchołek $y$ do~rozwiązania i~ustaw wartość zmiennej $x$ na~$y$.
    \item Jeżeli długość rozwiązania jest~mniejsza niż~oczekiwana, wróć do~punktu~2.
    \item Wróć do~wierzchołka początkowego.
\end{enumerate}

\subsection{Instancje problemu}

Problem komiwojażera jest~bardzo popularny i~możliwe jest~znalezienie wielu gotowych instancji, których~można użyć w~celu przeprowadzenia badań. W~naszych eksperymentach wybraliśmy 9~instancji z~udostępnionego przez~uniwersytet w~Heidelberg zbioru~\cite{instances}. Są~to:

\begin{enumerate}
    \item eil51,
    \item berlin52,
    \item pr76,
    \item kroA100,
    \item kroC100,
    \item rd100,
    \item lin105,
    \item ch130,
    \item tsp225.
\end{enumerate}

\noindent
Przy doborze konkretnych instancji zależało nam~na~tym, aby~nie~były~one zbyt duże (do~ok.~200~miast), a~także miały znalezione rozwiązania optymalne, były~różnorodne oraz~podane w~formacie EUC\_2D, czyli za~pomocą współrzędnych na~dwuwymiarowej płaszczyźnie Euklidesowej.