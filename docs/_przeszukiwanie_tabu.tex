\subsection{Opis}

Tabu Search jest drugim algorytmem optymalizacyjnym przedstawionym w sprawozdaniu, który pozwala na wychodzenie z optimów lokalnych poprzez dopuszczenie ruchów pogarszających rozwiązanie. Istnieją dwie popularne wersje tego rozwiązania --- z listą tabu, czyli zakazanych ruchów oraz z elitarną listą kandydatów. Na potrzeby eksperymentu zdecydowaliśmy się zaimplementować tę drugą wersję.

\subsection{Cechy algorytmu}

Jedną z głównych cech algorytmu i zarazem przyczyn jego powstania jest dopuszczenie wychodzenia z maksimów lokalnych. Algorytm ten pozwala także na ograniczenie się do przeszukiwania jedynie części sąsiadów, zamiast całego ich zbioru, co znacznie przyspiesza obliczenia. Jednak aby możliwe było pokonanie słabości algorytmów optymalizacji lokalnej, przeszukiwanie tabu naraża się na wpadanie w cykle. Trzeba też uważać, aby nie ograniczyć za bardzo jego ruchów.

\subsection{Elitarna lista kandydatów}

...

\subsection{Parametry}

...

\subsection{Podsumowanie}

...