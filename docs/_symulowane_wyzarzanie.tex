\subsection{Opis}

Algorytm symulowanego wyżarzania jest~heurystyczną metodą przeszukującą przestrzeń alternatywnych rozwiązań w~celu znalezienia najlepszego rozwiązania, czyli~takiego będącego najbliżej optimum pod~względem wartości funkcji kosztu. Polega on~na~iteracyjnym wyborze losowych rozwiązań z~przestrzeni lokalnych solucji i~akceptacji ich w~przypadku gdy~zmniejszają one~wartość funkcji kosztu, bądź~spełniają odpowiednie warunki probabilistyczne degradowane wraz z~upływem kolejnych iteracji. Algorytm opisujący kolejne kroki tej~metody został przedstawiony poniżej.

\begin{enumerate}
	\item Wygeneruj temperaturę startową $T_0$, rozwiązanie początkowe $x = x_{init}$, liczbę kroków $L$ co~które~aktualizowana jest~temperatura, oraz~stopień redukcji temperatury $\alpha~\epsilon~(0, 1)$.
	\item Powtarzaj dopóki warunek stopu nie~został spełniony:
	\begin{enumerate}
		\item Powtarzaj $L$ razy:
		\begin{enumerate}
			\item Generuj losowe rozwiązanie $x'$ w~sąsiedztwie rozwiązania $x$.
			\item Jeżeli nowe rozwiązanie jest~lepsze od poprzedniego $f(x') <= f(x)$, lub~warunek wyboru gorszego rozwiązania zostanie spełniony:
			\begin{enumerate}
				\item Zaakceptuj nowe rozwiązanie $x = x'$.
			\end{enumerate}
		\end{enumerate}
		\item Aktualizuj temperaturę $T_k = T_{k-1} * \alpha$.
	\end{enumerate}
\end{enumerate}

Powyżej przedstawiony generyczny algorytm może zostać wykorzystany w~wielu scenariuszach optymalizacyjnych. Dla~zadania postawionego w~tym zadaniu należało odpowiednio dostroić powyższy algorytm, tak~aby~najskuteczniej radził sobie w~znajdowaniu rozwiązania dla~problemu STSP.


\subsection{Dobór temperatury początkowej}
Jednym z~ważniejszych kroków w~procesie symulowanego wyżarzania jest~dobór odpowiedniej temperatury początkowej. Wartość ta~decyduje w~dużej mierze o~czasie zbieżności do~rozwiązania finalnego oraz~liczbie zaakceptowanych gorszych lokalnie rozwiązań, które~mogą prowadzić do~nieoczekiwanej poprawy jakości rozwiązania w~ujęciu globalnym. W~naszej implementacji zdecydowaliśmy się dobierać ten parametr w~sposób adaptatywny, ściśle zależny od~instancji problemu oraz~wygenerowanego pierwszego rozwiązania. Wartość ta~wyliczana jest~zgodnie z~poniższym wzorem.

\begin{equation}
	T_0 = \frac{\sum_{k=0}^{K} f(G(x)) - f(x)}{K * \ln \theta}
\end{equation}

W~powyższym wzorze, wektor $x$ jest~oryginalnym wygenerowanym rozwiązaniem, funkcja $G(x)$ generuje nowe rozwiązanie $x'$ w~sąsiedztwie rozwiązania $x$, a~funkcja $f(x)$ zwraca wartość kosztu danego rozwiązania. Parametr $K$ określa liczbę rozwiązań, które~chcemy wygenerować, a~$\theta$ procent akceptowanych ruchów. Tak~przedstawiona metoda pozwala na~lepszy dobór temperatury początkowej w~odniesieniu do~konkretnej instancji problemu. W~postaci przedstawionej powyżej jest~ona jednak wrażliwa na~jakość wygenerowanego rozwiązania początkowego i~może nie~zawsze być~prawidłowo dobrana. Wartość ta~może być~również obliczana rekurencyjnie biorąc pod~uwagę szersze spektrum rozwiązań co~pozwala na dokładniejszy dobór parametru \cite{saTemperature}. 


\subsection{Generowanie nowego rozwiązania}
W~każdej iteracji algorytmu symulowanego wyżarzania należy wybrać jedno losowe rozwiązanie z~przestrzeni lokalnych rozwiązań w~sąsiedztwie OPT-2. W~tym celu zdecydowaliśmy się wylosować dwie różne liczby całkowite w~przedziale $(0, n)$, gdzie~$n$ jest~rozmiarem wektora rozwiązania, a~następnie dokonać odwrócenia łuku między odpowiadającymi wierzchołkami w~wektorze rozwiązania. Dzięki takiemu podejściu, nie~jest~potrzebne marnotrawienie zasobów operacyjnych i~pamięciowych na~generowanie i~przechowywanie wszystkich rozwiązań dla~danej iteracji algorytmu, z~których następnie wybrane zostałoby tylko pojedyncze rozwiązanie, a~reszta byłaby zbędna.


\subsection{Warunek akceptacji rozwiązania}
Rozwiązanie $x'$ generowane w~każdej iteracji algorytmu zostaje zaakceptowane tylko w~przypadku, gdy~jest~ono nie~gorsze od~rozwiązania poprzedniego $f(x') <= f(x)$, lub~spełniony jest~warunek probabilistyczny:

\begin{equation}
	e^{-\frac{f(x') - f(x)}{T_k}} > random(0, 1)
\end{equation} 

Dzięki takiemu podejściu, pozwalamy algorytmowi wybrać rozwiązanie gorsze z~pewnym prawdopodobieństwem zależnym od~aktualnej wartości temperatury. Może to~mieć pozytywny skutek i~pozwolić nam wyjść z~lokalnego optimum, w~którym moglibyśmy utknąć gdybyśmy heurystycznie dążyli tylko ku~lokalnej poprawie. Należy zauważyć, iż~poprzez degradacje wartości temperatury w~kolejnych iteracjach, prawdopodobieństwo wybrania gorszego rozwiązania maleje.


\subsection{Warunek stopu}
Istotą każdego algorytmu optymalizacyjnego jest~dobrze zdefiniowany warunek stopu. W~naszej implementacji zdecydowaliśmy się kończyć przeszukiwanie przestrzeni rozwiązań, w~przypadku gdy~wartość temperatury $T_k$ spadła poniżej minimalnej wartości temperatury $T_\omega$, lub~gdy~w~przeciągu pewnej liczby iteracji $\sigma$ nie~udało się poprawić jakości rozwiązania. 


\subsection{Parametry}
Jakość algorytmu symulowanego wyżarzania zależy od~wielu parametrów, które decydują o~czasie zbieżności do~rozwiązania finalnego, liczbie przejrzanych lokalnie rozwiązań, czy~szansie na~akceptację rozwiązań gorszych. Metodą prób, błędów i~analizy dobraliśmy zestaw parametrów, które pozwalają w~relatywnie krótkim czasie znajdować rozwiązania bliskie optimum. Poniżej przedstawione zostały wartości parametrów przedstawionych w~tej sekcji.

\begin{itemize}
	\item[--] $\alpha = 0.9$ -- stopień redukcji temperatury w~kolejnych iteracjach.
	\item[--] $\theta = 0.95$ -- procent akceptowanych ruchów przy doborze temperatury początkowej.
	\item[--] $T_\omega = 0.1$ -- minimalna wartość temperatury kończąca algorytm.
	\item[--] $K = 1000$ -- liczba powtórzeń generowania rozwiązania przy~doborze temperatury.
\end{itemize}




