\subsection{Wnioski}

W problemie symetrycznego komiwojażera, stosunkowo łatwo znaleźć dość dobre rozwiązanie za pomocą prostej heurystyki, którą też można jeszcze udoskonalić.

Algorytmy przeszukiwania lokalnego przeważnie znajdują jednak jeszcze lepsze rozwiązania, nieraz nawet optymalne (przy instancjach o liczności ok. 50 miast), pracując wielokrotnie dłużej, jednak wciąż nie dłużej niż minutę dla wybranych instancji. Algorytmy przeszukiwania lokalnego mimo, że zaczynają ze stosunkowo słabym rozwiązaniem losowym, wciąż poszukując lepszych w jego sąsiedztwie, potrafią osiągać bardzo dobre wyniki.

Porównując algorytmy Greedy i Steepest, można zauważyć, że ostatecznie oba zwracają podobnej jakości rozwiązania. Steepest wykonuje kilkakrotnie mniej kroków, jednak w każdym z nim musi przeszukać całe sąsiedztwo aktualnego rozwiązania, w konsekwencji czego, sumarycznie, przeszukuje większą przestrzeń rozwiązań i trwa dłużej od algorytmu Greedy. Zaobserwowaliśmy, że ta tendencja jest odwrotna dla dla największej, wybranej instancji.

\subsection{Trudności}

Jednym z głównych problemów, jakie wystąpiły podczas prezentacji wyników, było odpowiednie dobranie wykresów. Mimo, że 3 z 4 prezentowanych algorytmów dawało zbliżone wyniki, to random zawsze znacznie się od nich różnił, przez co trudno było tak wyskalować wykresy, aby wyraźnie było widać wszystkie zależności. 

Innym problemem jest odpowiednie dobranie parametrów, aby każdy algorytm został uruchomiony odpowiednią liczbę razy dla każdej instancji, ale jednocześnie, by generowanie obliczenia nie trwały zbyt długo.

\subsection{Propozycje udoskonaleń}

Warta zbadania na pewno jest sytuacja zaobserwowana dla największej instancji, w której Steepest okazuje się być lepszy od algorytmu Greedy pod względem czasu wykonania, przegląda też mniejszą liczbę rozwiązań. Warto sprawdzić, czy ta tendencja utrzymuje się dla przykładów z większą liczbą miast.